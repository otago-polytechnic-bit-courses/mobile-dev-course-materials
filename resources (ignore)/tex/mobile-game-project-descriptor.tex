% Author: Grayson Orr
% Course: ID721001: Mobile Application Development

\documentclass{article}
\author{}
 
\usepackage{fontspec}
\setmainfont{Arial}

\usepackage{graphicx}
\usepackage{wrapfig}
\usepackage{enumerate}
\usepackage{hyperref}
\usepackage[margin = 2.25cm]{geometry}
\usepackage[table]{xcolor}
\usepackage{fancyhdr}
\hypersetup{
  colorlinks = true,
  urlcolor = blue
}
\setlength\parindent{0pt}
\pagestyle{fancy}
\fancyhf{}
\rhead{College of Engineering, Construction and Living Sciences\\Bachelor of Information Technology}
\lfoot{Project\\Version 3, Semester Two, 2024}
\rfoot{\thepage}
 
\begin{document} 

\begin{figure}
	\centering
	\includegraphics[width=50mm]{../img/logo.png}
\end{figure} 

\title{College of Engineering, Construction and Living Sciences\\Bachelor of Information Technology\\ID721001: Mobile Application Development\\Level 7, Credits 15\\\textbf{Project}}
\date{}
\maketitle

\section*{Assessment Overview}
In this \textbf{individual} assessment, you will develop \textbf{two} mobile games using \textbf{Unity} and publish them to \textbf{Google Play Store} or \textbf{Apple App Store}. Also, you will provide documentation that addresses several aspects of the mobile game development process. In addition to the mobile games and documentation, you will present the mobile games and answer follow up questions via a video recording.

\section*{Learning Outcomes}
At the successful completion of this course, learners will be able to:
\begin{enumerate}
	\item Implement and publish complete, non-trivial, industry-standard mobile applications following sound architectural and code-quality standards.
	\item Identify relevant use cases for a mobile computing scenario and incorporate them into an effective user experience design.
	\item Follow industry standard software engineering practice in the design of mobile applications.
\end{enumerate}

\section*{Assessments}
\renewcommand{\arraystretch}{1.5}
\begin{tabular}{|c|c|c|c|}
	\hline
	\textbf{Assessment}                                 & \textbf{Weighting} & \textbf{Due Date}            & \textbf{Learning Outcome} \\ \hline
	\small Practical & \small 20\%        & \small 13-11-2024 (Wednesday at 4.59 PM)   & \small 2, 3                   \\ \hline
	\small Project                 & \small 80\%        & \small 13-11-2024 (Wednesday at 4.59 PM) \small  & \small 1, 2, 3                   \\ \hline
\end{tabular}

\section*{Conditions of Assessment}
You will complete majority of this assessment during your learner-managed time. However, there will be time during class to discuss the requirements and your progress on this assessment. This assessment will need to be completed by \textbf{Wednesday, 13 November 2024} at \textbf{4.59 PM}.

\section*{Pass Criteria}
This assessment is criterion-referenced (CRA) with a cumulative pass mark of \textbf{50\%} over all assessments in \textbf{ID721001: Mobile Application Development}.

\section*{Authenticity}
All parts of your submitted assessment \textbf{must} be completely your work. Do your best to complete this assessment without using an \textbf{AI generative tool}. You need to demonstrate to the course lecturer that you can meet the learning outcome for this assessment. \\
 
 However, if you get stuck, you can use an \textbf{AI generative tool} to help you get unstuck, permitting you to acknowledge that you have used it. In the assessment's repository \textbf{README.md} file, please include what prompt(s) you provided to the \textbf{AI generative tool} and how you used the response(s) to help you with your work. It also applies to code snippets retrieved from \textbf{StackOverflow} and \textbf{GitHub}. \\
 
 Failure to do this may result in a mark of \textbf{zero} for this assessment.

\section*{Policy on Submissions, Extensions, Resubmissions and Resits}
The school's process concerning submissions, extensions, resubmissions and resits complies with \textbf{Otago Polytechnic} policies. Learners can view policies on the \textbf{Otago Polytechnic} website located at \href{https://www.op.ac.nz/about-us/governance-and-management/policies}{https://www.op.ac.nz/about-us/governance-and-management/policies}.

\section*{Submission}
You \textbf{must} submit all program files via \textbf{GitHub}. The latest program files in the \textbf{master} or \textbf{main} branch will be used to mark against the \textbf{Functionality} criterion. Please test your \textbf{master} or \textbf{main} branch application before you submit. Partial marks \textbf{will not} be given for incomplete functionality. Late submissions will incur a \textbf{10\% penalty per day}, rolling over at \textbf{5:00 PM}.

\section*{Extensions}
Familiarise yourself with the assessment due date. Extensions will \textbf{only} be granted if you are unable to complete the assessment by the due date because of \textbf{unforeseen circumstances outside your control}. The length of the extension granted will depend on the circumstances and \textbf{must} be negotiated with the course lecturer before the assessment due date. A medical certificate or support letter may be needed. Extensions will not be granted for poor time management or pressure of other assessments.

\section*{Resits}
Resits and reassessments \textbf{are not} applicable in \textbf{ID721001: Mobile Application Development}.

\section*{Instructions}
You will need to submit a mobile games and documentation that meet the following requirements:

\subsection*{Functionality - Learning Outcomes 1, 2, 3 (60\%)}
\begin{itemize}
	\item The mobile games needs to run without code or file structure modification in \textbf{Unity}.
	\item Playable on a variety of mobile devices, i.e., devices with different screen sizes.
	\item Free of bugs that significantly affect the playability.
	\item The mobile games are published to \textbf{Google Play Store} or \textbf{Apple App Store}.To published to \textbf{Google Play Store} or \textbf{Apple App Store}, you will need an account. The account's credentials will be privately given to you on \textbf{Microsoft Teams}. \textbf{Do not} disable any applications published on this account.
	\item Ability to download the mobile games from \textbf{Google Play Store} or \textbf{Apple App Store} on to a variety of mobile devices.
\end{itemize}

\subsection*{Documentation - Learning Outcomes 2, 3 (20\%)}
For each game, in a \textbf{Microsoft Word} document, explain the following:
\begin{itemize}
	\item Core concept.
	\item Inspiration games.
	\item Design pillars.
	\item Main features and mechanics.
	\item Target platform and audience.
	\item Interface and controls.
	\item Basic story.
	\item Visual style.
	\item Initial design sketches. These can be hand-drawn or created using a design tool.
	\item Audio style.
	\item Known issues and bugs.
	\item Future improvements.
	\item A URL to the games on \textbf{Google Play Store} or \textbf{Apple App Store}.
	\item A URL to your presentation on your \textbf{Microsoft OneDrive}.
\end{itemize}

\subsection*{Presentation - Learning Outcome 3 (15\%)} 
\begin{itemize}
	\item Present the mobile games via a video recording. In addition, you need to answer the following:
	\begin{itemize}
		\item How did you plan and prioritise features throughout the development process?
		\item What tools and technologies did you utilise to streamline your development workflow?
		\item How did you handle potential challenges, such as time management and motivation?
		\item What strategies did you employ to maintain code quality and avoid technical debt during the development process?
		\item How did you handle testing and debugging of the mobile games?
	\end{itemize}
\end{itemize}

\subsection*{Git Usage - Learning Outcome 3 (5\%)}
\begin{itemize}
	\item A \textbf{GitHub} project board or issues to help you organise and prioritise your development work. The course lecturer needs to see consistent use of the \textbf{GitHub} project board or issues for the duration of the assessment.
    \item Your \textbf{Git commit messages} should:
    \begin{itemize}
      \item Reflect the context of each functional requirement change.
      \item Be formatted using an appropriate naming convention style.
    \end{itemize}
\end{itemize}

\subsection*{Additional Information}
\begin{itemize}
    \item \textbf{Do not} rewrite your \textbf{Git} history. It is important that the course lecturer can see how you worked on your assessment over time. 
    \item You need to show the course lecturer the initial \textbf{GitHub} project board or issues before you start your development work. Following this, you need to show the course lecturer your \textbf{GitHub} project board or issues at the end of each week.
\end{itemize} 

\end{document}